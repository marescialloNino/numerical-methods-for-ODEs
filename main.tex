\documentclass[a4paper,12pt]{article}

% Packages
\usepackage[utf8]{inputenc}
\usepackage{amsmath}
\usepackage{graphicx}
\usepackage{hyperref}
\usepackage{geometry}
\usepackage{float}

% Page margins
\geometry{left=3cm, right=3cm, top=2cm, bottom=2cm}

% Document settings
\title{Numerical Methods Homework 1}
\author{Nicola Schiavo}
\date{\today}

\begin{document}

\maketitle

\begin{abstract}
In this document, we present the numerical solution of various exercises, showcasing the application of numerical methods to solve ordinary differential equations (ODEs) that model different physical phenomena or mathematical problems. Each exercise will include a problem statement, the methodology applied, and the results obtained, followed by a brief discussion.
\end{abstract}

\section{Exercise 1}

"We consider a dynamical system governed by the differential equation $\frac{dy}{dt} = f(y, t)$, where the function $f$ will be specified. The initial condition is given by $y(0) = y_0$."

"To solve the above equation, we will implement the Euler method, a simple yet powerful numerical procedure. Given a step size $h$, the method updates the solution using the iterative scheme $y_{n+1} = y_n + h f(y_n, t_n)$."

"After implementing the Euler method with a step size of $h = 0.01$ over the interval $[0, 1]$, we observed the following behavior..."

% Include a figure (Make sure to include the actual file in your LaTeX project)


"The obtained numerical solution shows a consistent convergence to the expected analytical solution, confirming the reliability of the Euler method for this type of problem..."


% Uncomment the following two lines if you have references
%\bibliography{your_bib_file}
%\bibliographystyle{ieeetr}

\end{document}